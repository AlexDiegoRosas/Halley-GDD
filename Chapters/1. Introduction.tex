\section{Configuraciones y personajes}
\subsection{Mundo del juego}
\subsubsection{Apariencia general del mundo}

El mundo del juego se divide en 3 escenarios, uno dentro de
una casa especificamente el el pasillo o pasadiso, el otro esta
ubicado en la calle de la casa en una ciudad, para finalmente
el ultimo lugar esta ubicado en un bosque a las afueras de la
ciudad. En general el ambiente es oscuro en la noche a la luz de la luna

\begin{itemize}
	\item Área 1
	      \begin{itemize}
		      \item Descripción general: El pasillo de una casa, con luces tenues, con
		            objetos alrededor, unos cuadros que cuelgan de las paredes, como si fueran
		            otros mundos.
		      \item Características físicas: De apariencia moderna pero con un toque
		            clasico manteniendo algunas cosas añejas, con puertas de maderas, luces, etc.
		      \item Nivel que aparecen en dicha área: El protagonista aparece ahí en la
		            primera escena como parte del nivel 1 de Run or Wake.
		      \item Conexiones hacia otras áreas: Conexión al Área 2.
	      \end{itemize}
	\item Área 2
	      \begin{itemize}
		      \item Descripción general: Una calle oscura a la luz de la luna.
		      \item Características físicas: Tachos de basura, hidrantes, postes de luz, basura,
		            cilindros, pavimento pintado.
		      \item Conexiones que aparecen en dicha área: El protagonista aparece en el nivel 2
		      \item Conexiones hacia otras áreas: Conexión al Área 3.
	      \end{itemize}
	\item Área 3
	      \begin{itemize}
		      \item Descripción general: Un bosque frondoso, con el sonido de la fauna
		            que habita en ella.
		      \item Características físicas: Árboles, arbustos, vegetación, búhos, ramas caídas, hojas
		            caídas.
		      \item Niveles que aparecen en dicha área: Aparece el el nivel 3.
		      \item Conexiones hacia otras áreas: Conexión al Área 1
	      \end{itemize}
\end{itemize}

\subsubsection{Personajes}
\begin{itemize}
	\item Personaje 1 (protagonista)
	      \begin{itemize}
		      \item Transfondo: Un niño que viaja a través de sus pesadillas
		      \item Personalidad: Es un niño miedoso, asustadiso, temeroso.
		      \item Vista:
		            \begin{itemize}
			            \item Características físicas:
			                  \begin{itemize}
				                  \item Cabello negro
				                  \item Trigueño
				                  \item Rasgos andinos
				                  \item Viste una pijama
			                  \end{itemize}
			            \item Animaciones
			                  \begin{itemize}
				                  \item Saltar
				                  \item Correr
				                  \item Desplazamiento derecho - izquierda
				                  \item Agacharse o rodar
				                  \item Susto
				                  \item Muerte
			                  \end{itemize}
		            \end{itemize}
		      \item Habilidades especiales
		            \begin{itemize}
			            \item La velocidad aumenta con el tiempo
		            \end{itemize}
		      \item Relevancia con la historia del juego
		            \begin{itemize}
			            \item Escapando de sus pesadillas
		            \end{itemize}
		      \item Relación con otros juegos
		            \begin{itemize}
			            \item Usamos como referencia Subway Surfers
		            \end{itemize}
	      \end{itemize}
	\item Personaje 2 (Enemigo - pesadilla - mounstruo)
	      \begin{itemize}
		      \item Transfondo: Un mounstruo que intenta atrapar a un niño en sus pesadillas
		      \item Personalidad: Malvado, frio, impulsivo.
		      \item Vista
		            \begin{itemize}
			            \item Características físicas
			                  \begin{itemize}
				                  \item Humanoide
				                  \item Caricaturesco
			                  \end{itemize}
			            \item Animaciones
			                  \begin{itemize}
				                  \item Saltar
				                  \item Correr
				                  \item Desplazamiento derecho - izquierda
				                  \item Agacharse o rodar
				                  \item Atrapar a niño o agarrar
			                  \end{itemize}
		            \end{itemize}
	      \end{itemize}
\end{itemize}

\section{Niveles}
\subsection{Nivel 1}
\subsubsection{Sinopsis}
\begin{itemize}
	\item Material introductorio (¿corte de escena? ¿instrucciones de la misión?):
	      Correrá en linea recta hasta donde pueda.
	\item Objetivos:
	      El mapa debe ser interesante y muy visltoso, además que tebdrá que ser un reto
	      para el jugador.
	\item Descripción física:
	      Será 3 niveles es decir 3 escenarios.
	\item Mapa:
	      Constará de Escenariosn: Un pasillo de una casa con luces y ambientación tétrica.
	      La calle que estará rodeado de basura, hidrantes, postes de luz y demás cosas, por
	      último estará el bosque. este tendrá árboles, arbustos, ramas caídas, hojas, búhos y
	      sonidos de animales.
	\item Camino principal:
	      Tendrá que ir en linea reacta hasta donde el jugador pueda llegar.
	\item Encuentros:
	      No habrá encuentros, pero el enemigo estará persiguiendo constantemente al protagonista.
	\item Walkthrough del nivel:
	      El nivel acabara cuando el personaje pierda.

\end{itemize}

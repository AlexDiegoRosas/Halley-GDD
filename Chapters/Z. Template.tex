\section{Historia}\pagenumbering{arabic}

Había una vez un valiente niño llamado Wilfredo que tenía una gran
imaginación. Sin embargo, cuando llegaba la hora de dormir, se
enfrentaba a un desafío, las pesadillas. No importa cuán cansado
estuviera, su mente se llenaba de temores y preocupaciones que se
manifestaban en sus sueños.\\

Una noche, Wilfredo se encontró en un mundo onírico, donde un monstruo
feroz lo perseguía sin descanso. El terror se apoderaba de él mientras
trataba de escapar de la criatura aterradora. Sus pies pequeños se movían
velozmente, su corazón latía con fuerza y su respiración se aceleraba. Era
una carrera contra el miedo.\\

Pero a medida que corría, Wilfredo comenzó a darse cuenta de algo increíble.
Su valentía y determinación le daban fuerzas para superar cualquier obstáculo
en su camino.

Mientras escapaba del monstruo en su sueño, Wilfredo descubrió que tenía el
poder de enfrentar sus miedos y convertirlos en oportunidades para crecer.
Su confianza creció y, con cada paso, el miedo comenzó a disiparse.\\

Finalmente, en lo más profundo de su aventura onírica. Wilfredo encontró la clave
para derrotar al monstruo. No era la fuerza ni la violencia, sino el amor y la
compasión. Extendió la mano hacia la criatura y, con una sonrisa, le mostró que no
había necesidad de temer.\\

En ese momento el monstruo se transformó en una figura amigable y tierna. Era un
reflejo de los miedos internos de Wilfredo, pero ahora era un símbolo de
amistad y superación. Juntos, emprendieron un viaje lleno de aventuras y aprendizaje.\\


Continuará...

\section{Información general del juego}

\subsection{Concepto del juego}

Run or Wake es un juego runner de corta duración disponible para PC desde los mas bajos recursos de memoria, cpu y gpu. un juego sencillo y entretenido.

\subsection{Características}

\begin{itemize}
	\item 3 niveles
	\item Diseño 3D \& LowPoly
	\item La historia es de carácter onírico
	\item Offline (sin conexión a Internet)
	\item Se juega en PC de escritorio
	\item Contiene reglas especificas
\end{itemize}

\subsection{Género}

Run or Wake pertenece al genero endless runner. A continuación se
da un breve concepto de lo que es este género.

\subsubsection{ Endless runner}
El género (endless runner) es un tipo de videojuego en el que el
personaje principal corre de forma continua a través de un entorno
en constante cambio y el objetivo es sobrevivir el mayor tiempo
posible mientras se evitan obstáculos y recolectan objetos.

\subsection{Audiencia objetivo}

Run or Wake tiene como púbico objetivo a jugadores de un amplio rango
de edades con un tiempo corto que dedicar al ocio electrónico. Por tal
motivo apostamos por el público mayor a los 12 años.

\subsection{Game flow}

Cada nivel de Run or Wake ofrece la posibilidad de cumplir con un objetivo,
siendo la recolección de una cantidad fija de estrellas, mientras
esquivas obstáculos con movimientos sencillos siendo perseguido por
un monstruo. Para ello nos valdremos de los siguientes elementos.

\begin{itemize}
	\item \textbf{Movilidad}: En Run or Wake controlamos un personaje
	      con los movimientos por defecto de unity las flechas y las teclas A,D.
	\item \textbf{Puntos fuertes y débiles}: En Run or Wake los puntos
	      débiles para el protagonista son los obstáculos que se encontrará
	      en el camino, el punto fuerte sería la velocidad plus que adquiere
	      al tocar objetos que tienen esa capacidad de aumentarle la energía
	      para mas velocidad.
	\item \textbf{Mejoras}: El jugador deberá de recolectar objetos para
	      adquirir distintas mejoras como aumentarle la velocidad o nivel de
	      vida.
\end{itemize}

\subsection{Apariencia del juego}
\subsubsection{Estilo visual}
Run or Wake tendrá en estilo sencillo, no demasiado detallista para encajar
con su carácter amigable. El estilo visual que más encaja con este concepto
es el de LowPoly. Los personajes serán caricaturescos con colores vivos y
texturas simples.

\subsubsection{Experiencia del juego}

Run or Wake ofrece una Experiencia agradable a la vista sin mucho ruido que
pueda distraer de cumplir con sus objetivos apostando por un juego capaz de
hacer que el usuario se divierta y entretenga.

\subsection{Alcance del proyecto}

\begin{itemize}
	\item Número de localizaciones: 3 escenarios
	\item Número de niveles: 3 niveles
	\item Número de NPCs (Non Player Characters o Personaje No Controlado): 1 Enemigo
	\item Objetivo: Recolectar 30 almuadas nivel 1, 40 en el nivel 2 y finalmente 60
	      almuadas para el nivel 3 y final
\end{itemize}
